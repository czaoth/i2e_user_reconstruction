%!TEX root = ../paper.tex
\newcommand{\concept}{\textit{Konzept: }}
\newcommand{\currentState}{\textit{Aktueller Stand: }}
\newcommand{\openQuestions}{\textit{Nächste Schritte: }}

% components
\newcommand{\visualization}{Visualisierung }
\newcommand{\constructionLogic}{Konstruktionslogik }
\newcommand{\gestureRecognition}{Natural User Interface }
\newcommand{\mobileDeviceComponent}{Mobile Device }
\newcommand{\netWorkMiddleware}{Netzwerk Middleware }
\newcommand{\userReconstruction}{User Reconstruction }

% common terminology
\newcommand{\mobileDevice}{See-Through AR-Brille }
\newcommand{\physicalGestures}{Physikalische Gesten }
\newcommand{\interpretedGestures}{Interpretierte Gesten }

\section{Architektur} \label{sec:architecture}

Wir verfolgen mit unserer Architektur eine lose Kopplung der Komponenten und 
effiziente Skalierung mit der verfügbaren Hardware. Dies erlaubt eine einfache
Erweiterbarkeit und den problemlosen Austausch von Technologien. Somit können
wir schnell und effizient auf veränderte Anforderungen während der Entwicklung
reagieren.\\

Die kollaborative verteilte Umgebung besteht aus mehreren Client Instanzen. Eine
Client Instanz ist als Umgebung definiert, die an jedem Standort ausgeführt
wird. Jede Client Instanz besteht aus mehreren Komponenten, die über das
Netzwerk mithilfe der \netWorkMiddleware kommunizieren.
Abbildung ~\ref{fig:flowchart} zeigt die Abhängigkeiten, angebundenen Sensoren
und übergeordneten Datenfluss.

Die \userReconstruction ist für die online 3D Rekonstruktion von Benutzern
zuständig. Das Ergebnis, ein Mesh des Benutzers, wird an alle anderen Client
Instanzen für Darstellungszwecke übertragen.

Die \gestureRecognition kümmert sich um die Eingaben durch den lokalen Benutzer.
Die \constructionLogic wird kontinuierlich von der \gestureRecognition über die
Bewegungen der Arme und Hände des Benutzers informiert. Dadurch ist es möglich
virtuelle Objekte direkt zu greifen, anzuheben oder zu verschieben. Wir nennen
dies \physicalGestures.

Die \gestureRecognition erkennt außerdem Gesten des Benutzers und leitet diese
an die \constructionLogic weiter. Wir nennen dies \interpretedGestures. Diese
bildet die Gesten auf Aktionen wie bspw. das zusammenfügen oder
auseinandernehmen von virtuellen Objekten ab.

Die \constructionLogic stellt sicher, dass nur Transformationen auf dem
unterliegende Modell ausgeführt werden können, die dieses von einem konsistenten
Zustand in einen konsistenten Zustand überführen. Des weiteren löst die
\constructionLogic auch Konflikte zwischen in Konflikt stehenden Aktionen
verschiedener Benutzer auf.

Die \visualization erstellt aus dem Modell der \constructionLogic eine virtuelle
Szene. Die \mobileDeviceComponent informiert die \visualization über den
viewport aus der diese Szene für die jeweilige \mobileDevice gerendered werden
muss. Die Ergebnisse der Render-Prozesse werden dann an die
\mobileDeviceComponent weitergeleitet wird.

\todo[inline]{Grafik übersetzen}
\begin{figure}[H]
	\includegraphics[width=\columnwidth]{figs/FlowChart_v2}
	\caption{Client Architektur}
	\label{fig:flowchart}
\end{figure}

Das Projekt Arvida \cite{Arvida2015} verfolgt einen ganz ähnlichen Ansatz. Die
Hauptunterschiede bestehen in der hohen Fokussierung auf Webtechnologien und die
Einbindung von Diensten dritter. Unsere Architektur implementiert eine Teilmenge
der in \cite{Arvida2015} beschriebenen Funktionalitäten. 

\section{Netzwerk Middleware}
\label{sec:middleware}
Unser Ziel ist eine einfach erweiterbare Architektur deren Komponenten beliebige
Programmiersprachen verwenden und auf beliebigen Betriebssystemen ausgeführt
werden können. Für diese Zielstellung ist eine Middleware wünschenswert, die
möglichst weit von der unterliegenden Netzwerktechnik abstrahiert.

Wir möchten objektorientiert, Nachrichten basiert, mit einem geringen overhead
und Typsicher über das Netzwerk kommunizieren. Die von uns entwickelte
Netzwerk-Middleware erfüllt diese Kriterien.

Für eine performante Serialisierung der Daten wird Protobuf verwendet. Für jede
verwendete Sprache muss eine eigene Implementation der Middleware
geschrieben werden. Um die Kompatiblität der Implementationen untereinander zu
gewährleisten halten sich jede Implementation
an den von Google vorgegebenen Wire-Standard für Protobuf-Nachrichten.

Protobuf-Nachrichten sind nicht selbst identifizierend. Dewegen arbeiten wir mit
einer Text-Datei, die eine Abbildung von Identifiern auf Nachrichtentypen zur
Verfügung stellt. Diese wird automatisch beim kompilieren der
Protobuf-Nachrichten Definitionen erstellt.

Damit sichergestellt ist, dass alle beteiligten Systeme die selben
Nachrichtendefinitionen und Identifier nutzen sind alle Protobuf
Nachrichtendefinitionen in einem zentralen Git-Repository abgelegt.

\section{\constructionLogic} \label{sec:constructionlogic}

Die \constructionLogic hält den aktuellen Zustand des Modells einer
Client-Instanz. Auf allen Client-Instanzen wird die selbe Logik ausgeführt um
den Zustand des Modells abzufragen und zu manipulieren. Das Modell einer
Client-Instanz wird dabei für die Synchronisierung mit den anderen Instanzen als
Master festgelegt. Differiert der Modellzustand einer Client-Instanz mit dem dem
Master wird diese an den Zustand des Master-Modells angepasst. Um Konflikte
zwischen den Aktionen der Benutzer zu erkennen und zu jedem Zeitpunkt einen
konsistenten Modellzustand zu gewährleisten werden alle Aktionen von Benutzern
auf Actions abgebildet.

Jede Action unterstützt die Abfrage, ob die Aktion im derzeitigen Modellkontext
ausgeführt werden kann und bietet die Möglichkeit die verursachten Änderungen
rückgängig zu machen (rollback). Außerdem legt jede Aktion fest, wie lange ein
Protokoll der Aktion aufbewahrt werden soll. Eine Aktion kann nur dann
ausgeführt werden, wenn sie im lokalen Modellkontext gültig ist. Jede Aktion
kann lokal ausgeführt werden, muss aber von der Master-Instanz bestätigt werden
um für alle Client-Instanzen Gültigkeit zu erlangen. Alle so validierten
Aktionen werden für einen von der Aktion abhängigen Zeitraum protokolliert.
Falls die Aktion im Modell Kontext der Master-Instanz nicht gültig ist wird
diese in der jeweiligen Client-Instanz rückgängig gemacht.

Der Zustand des Modells kann nur durch Aktionen von Benutzern geändert werden.
Falls also eine Aktion im Kontext der Master-Instanz keine Gültigkeit hat kann
dies nur durch die Aktion eines anderen Benutzers ausgelöst worden sein. Ein
Konflikt wurde erkannt. Im einfachsten Fall wird nun die (nun invalide) Aktion
rückgängig gemacht. Je nach Domäne kann aber auch versucht werden die
Benutzeraktionen zusammen zu führen, da genug wissen vorhanden ist um die
spezifische konkurrierende Aktion genau zu bestimmen.\\

Die Constraints wurden für einen einfachen Anwendungsfall
implementiert und erfolgreich getestet. Wir haben uns als ersten Anwendungsfall
für die Konstruktion einer Murmelbahn entschieden, weil dafür nur ein sehr
begrenztes Domänenwissen notwendig ist.

\section{\visualization}
\label{sec:visualization}
Damit die Benutzer den aktuellen Modellzustand einfach wahrnehmen können
benötigen wir eine visuelle Repräsentation. Diese wird von der \visualization
aus dem aktuellen Modellzustand erzeugt. Für jeden Benutzer wird diese
Repräsentation aus seinem aktuellen Sichtbereich gerendered. Diese Daten werden
dann an die jeweilige \mobileDevice weitergeschickt.\\

\subsection{\gestureRecognition Anbindung}

Wir möchten virtuelle Objekte anfassen und die Anwendung durch Gesten steuern
können. Um virtuelle Objekte physikalisch manipulieren zu können sendet die
\constructionLogic die Position, Geschwindigkeit, Beschleunigung und Rotation
der Arme, Hände und Finger an die \constructionLogic. Gesten des Benutzers
werden von der \gestureRecognition interpretiert und auf abstrakte Befehle
abgebildet. Derzeit arbeiten wir an der physikalischen Interaktion mit
virtuellen Objekten.\\ 

\subsection{\mobileDeviceComponent Anbindung}

Um eine möglichst intuitive Wahrnehmung des Modells der \constructionLogic zu
ermöglichen geben wir das gerenderte Ergebnis der \visualization auf einer
\mobileDevice aus die vom Benutzer getragen wird. Die \mobileDeviceComponent
gibt an aus welcher Position die jeweiligen Benutzer die virtuelle Szene sehen.
Dafür werden Position und Rotation der jeweiligen \mobileDevice an die
\visualization geschickt. Die virtuelle Szene wird dann aus der jeweiligen
Position gerendered und das Ergebnis zurück an die Brille geschickt.

\subsection{\userReconstruction Anbindung}

Wir glauben, dass Benutzer intuitiv mit anderen Benutzern in Konflikt stehende
Aktionen vermeiden, wenn sie diese möglichst realitätsnah visualisiert werden.
Zu diesem Zweck schickt die \userReconstruction das texturierte Mesh des lokalen
Benutzers an die \visualization welche dies an alle anderen Client Instanzen
verteilt. Die Meshes der Benutzer der anderen Client Instanzen werden von der
lokalen \visualization empfangen und dargestellt.

\noindent\makebox[\linewidth]{\rule{\paperwidth}{0.4pt}}
\todo[inline]{Nächste Schritte in einem eigenen Abschnitt zusammen fassen.}

Ein Fehler der häufig auftritt sind differierende
Netzwerknachrichtendefinitionen auf den beteiligten Systemen. Um
den Fehler sofort nach dem Verbindungsaufbau zu bemerken ist die Verwendung
einer Checksumme der verwendeten Nachrichtendefinitionen geplant.

Die Constraints werden als nächstes für komplexere
Anwendungsfälle aus dem Bereich Maschinenbau implementiert und getestet werden.

Die Position der Hände und Finger wird derzeit unmodifiziert von der
\constructionLogic übernommen. Dies erfordert eine umsichtige Bedienung durch
den Benutzer. Bspw. ist es möglich in virtuelle Objekte hinein zu greifen. Dies
führt zu Fehlern in der verwendeten Physik-Engine. In einem nächsten Schritt
werden die Positionen entsprechend angepasst werden um diese Fehler zu vermeiden
und die Bedienung zu vereinfachen.

\noindent\makebox[\linewidth]{\rule{\paperwidth}{0.4pt}}

